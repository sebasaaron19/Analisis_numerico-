\documentclass{article}
\usepackage[utf8]{inputenc}

\title{Análisis Numérico}
\author{Aaron Sebastian Castillo Espinoza }
\date{Semestre 2019-2}

\begin{document}

\maketitle
\section{Bisección}
Este método en generar una sucesión de la siguiente manera:\\
\begin{verbatim}
   Inicio
     -Obtener a,b tales f(x) sea continua en [a,b]
      y que f(a)f(b)<0
     -Repetir el siguiente procedimiento
      hasta que se cumpla el criterio de paro 
      {
         x <- (a+b)/2
         imprimir x
         si f(a)f(x)<0 entonces
         b <- x
         sino 
         a <- x
      }   
   Fin   
\end{verbatim} 

Criterios de paro\\ 
\begin{enumerate}
    \item Hasta que $f(x)=0$ (Criterio fuerte), se nota mucho cuando la derivada es grande cerca de ka raíz. 
    \item Hasta que $|f(x)| <eps, eps\ge0 $ dado por el usuario 
    \item cantidad de iteraciones dadas por el usuario.
    \item Dado $\epsilon$ épsilon calcular n, donde n=Techo($\frac{ln(\frac {b-a}{\epsilon})}{ln(2)}$) 
\end{enumerate}
\end{document}
